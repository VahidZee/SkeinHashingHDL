\section{\textbf{نحوه‌ی استفاده از مدل طلایی}}
برای اجرای کد مدل طلایی، تابع \lr{skeinhash} صدا شده‌است. ورودی این تابع، یک متغیر از جنس  \lr{constant void*} است و آدرس ذخیره‌ی ‌خروجی توابع هش هم همراه با ورودی به تابع پاس داده شده‌است. برای این کار یک اسکریپت \href{https://github.com/VahidZee/SkeinHashingHDL/blob/master/SourceCode/C/main.c}{\lr{{main.c}}}  و یک اسکریپت \href{https://github.com/VahidZee/SkeinHashingHDL/blob/master/SourceCode/C/skeinhash.h}{\lr{{skeinhash.h}}}  به کد اضافه شده‌است. در اسکریپت \lr{main.c} متغیر ورودی، یعنی دیتای ‌تهیه شده برای تست عملیات هش  با نام \lr{input}و آدرس ذخیره‌ی ‌\lr{output} تعریف شده‌اند. در تابع \lr{skeinhash} منغیری که نشان‌دهنده‌ی طول \lr{input} است،  ۸۰ بایت در نظر گرفته شده، با این حال کد مدل طلایی برای هر طولی پاسخ‌گو است. متغیر ‌ورودی در  \lr{main.c} برای طول ۸۰ از جنس رشته (‌آرایه‌ای ۸۰تایی از ‌ \lr{char}) تعریف شده‌‌، تا مقداردهی آن بایت به بایت انجام شود. خروجی خود تابع‌های داخلی هش، ۶۴ بایت یا ۵۱۲ بیت است، اما در نهایت ۳۲ بایت اول این خروجی در \lr{output} ریخته شده‌اند، در نتیجه در ‌\lr{main.c} این متغیر ۳۲ بایتی تعریف شده‌است. در نهایت این خروجی در مبنای ۱۶ چاپ شده‌است.