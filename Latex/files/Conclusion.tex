\chapter{نتیجه‌گیری}


در بخش اول مقاله مقدمه‌ای در رابطه با \lr{Skein hashing} تهیه شده‌است. در این بخش توضیح کاملی درباره‌ی الگوریتم، توابع داخلی و فرمول‌های استفاده شده در این هش داده شده‌است.  \\
در ادامه، بخش دو و سه‌ی مقاله در رابطه با مستندسازی کدها آورده شده‌اند. در این مستندساز‌ی‌ها علاوه بر توضیح دقیق کد، به توضیح علت طراحی هر قطعه کد با توجه به استفاده‌ی آن در الگوریتم  \lr{Skein hashing} نیز پرداخته شده‌است. \\ در بخش دوم مقاله،‌ به مستندسازی بخش پیاده‌سازی سخت‌افزاری با وریلاگ پرداخته شده‌است. در زیربخش اول این قسمت،  اشتباهات منطقی یا سینتکسی موجود در کد و راه حل ارائه شده برای هر یک از آن‌ها و سپس مستندسازی قسمت به قسمت ماژول‌های وریلاگ آورده‌ شده‌است.‌ سپس در زیربخش جمع‌بندی، نمودار درختی این ماژول‌ها برای توصیف ساختار کلی طراحی سخت‌افزاری طراحی شده‌است. در انتها زیربخش‌های شبیه‌سازی و توضیحات مربوط به تست‌بنچ قرار گرفته‌اند. در این دو قسمت اطلاعات مربوط به شبیه‌سازی نظیر کلاک‌ها، تصویر موج‌ها و توصیفی از ساختار تست‌بنچ آورده شده‌اند.  \\ بخش سه مربوط به مستندسازی مدل طلایی است. در این بخش ابتدا توضیح مختصری در رابطه با دلیل استفاده از مدل طلایی و نموداری درختی از ساختار کلی مدل طلایی، سپس ساختارها و توابع و پس از آن روند اجرا و استفاده از کد مدل طلایی توصیف شده‌اند. \\هم‌چنین بخشی هم مربوط به سنتز و اطلاعات مربوط به آن نوشته شده‌است. \\
در این‌ پروژه لزوم استفاده از مدل طلایی برای مقایسه‌ و اطمینان حاصل کردن از صحت کد سطح پایین‌تر در کنار طراحی تست، نحوه‌ی مستندسازی و لزوم استفاده از آن برای قابل فهم کردن  و درنتیجه ارتقا دادن کد، نحوه‌ی شبیه‌سازی و سنتز و اصلاح یک برنامه در مقیاسی مشابه با صنعت آموخته‌شد.\\
همچنین منبع استفاده شده در این پروژه را \hyperref[manba]{اینجا} ببینید.

