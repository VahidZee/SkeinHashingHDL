\chapter{معرفی الگوریتم
\lr{Skein}
}
\section{
مقدمه
}
دردنیای امروز، با افزایش لحظه‌ای اطلاعات در جهان، روز‌ به ‌روز رمزنگاری و رمزگذاری اطلاعات اهمیت دوچندانی پیدا می‌کند.برای مثال برقراری امنیت سیستم‌ها و شبکه‌های رایانه‌ای، ذخیره‌ی اطلاعات مهم و حساس و ... همگی مثال‌هایی هستند که بدون رمزنگاری و رمزگذاری ممکن نخواهند بود. بدون اغراق، بدون رمزگذاری و رمزنگاری، اینترنت به شکلی که امروز وجود دارد به هیچ عنوان وجود نمی‌داشت.
راه‌های بسیار متفاوتی برای رمزنگاری و رمزگذاری وجود دارد، یکی از مهم‌ترینِ آنها، استفاده از توابع رمزگذاریِ بر پایه ی
\textit{ 
	درهم‌سازی (
\lr{Hashing}
)
}
می باشد. درهم‌سازی به خودی خود پرکاربرد ترین داده‌ساختار استفاده شده در علوم رایانه‌ای است. برخی از توابع درهم‌سازی شامل ویژگی‌هایی هستند که آنهارا برای استفاده برای کاربرد‌هایی چون رمزنگاری بسیار مناسب می‌کند. مهم‌ترین و پرکاربردترین این توابع، توابعی از خانواده‌ی 
\textit{\lr{SHA}}
یا 
\textit{\lr{Secure Hashing Algorithms}}
می‌باشند، توابعی چون 
\lr{MD-5}
،
\lr{SHA-0}
،
\lr{SHA-1}
،
\lr{SHA-2}
و 
\lr{SHA-3}
که هر کدام خود شامل خانواده ای از توابع مخصوص کاربرد‌های خاص خود هستند.

توابع خانواده ی 
\lr{SHA}
توسط گروهی به نام 
\textit{\lr{NIST}}
که کوتاه شده‌ی عبارت
\textit{\lr{National Institute of Standards and Technology}}
است، برگزیده می‌شوند. برای انتخاب هریک از ورژن های توابع این خانواده، ابتدا توابعی پیشنهاد شده، پس از آن تعدادی از آنها به عنوان فینالیست توسط 
\lr{NIST}
اعلام شده و در نهایت از بین فینالیست‌ها، یک تابع به عنوان ورژن جدید از توابع 
\lr{SHA}
معرفی می‌شود.

	تابع مورد بررسی در این مقاله یکی از توابع فینالیست برای انتخاب 
	\lr{SHA-3}
	می‌باشد که 
	\lr{Skein}
	نام دارد. 
	
\section{
معرفی اجمالی الگوریتم
}
\subsection{
انواع
}
\section{
روند اجرایی الگوریتم
}



	